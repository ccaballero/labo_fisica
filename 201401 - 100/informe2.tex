%\documentclass[letter,11pt]{article}
\documentclass[letter,twoside,11pt]{article}

\usepackage[spanish]{babel}
\usepackage[utf8]{inputenc}

\usepackage{lmodern}
\usepackage[T1]{fontenc}
\usepackage{textcomp}

\usepackage{graphicx}
\usepackage{pstricks}

\usepackage{anysize}
\marginsize{3cm}{2cm}{2cm}{3cm}

\usepackage{amsmath}
\usepackage{array}

\usepackage{fancyhdr}
\usepackage{lastpage}
\pagestyle{fancy}
\fancyhf{}
\fancyhead[LE,RO]{Laboratorio de Física Básica I}
\fancyfoot[CO,CE]{\thepage\ de \pageref{LastPage}}

\special{papersize=215.9mm,279.4mm}

\newcommand{\blankpage}{
\newpage
\thispagestyle{empty}
\mbox{}
\newpage
}

\renewcommand{\arraystretch}{1.2}

\begin{document}

\begin{titlepage}
\begin{center}
{\Large UNIVERSIDAD MAYOR DE SAN SIMÓN}\\
\vspace*{0.15cm}
{\large FACULTAD DE CIENCIAS Y TECNOLOGÍA}\\
\vspace*{0.10cm}
DEPARTAMENTO DE FÍSICA\\
\vspace*{3.0cm}
{\Large \textbf{LABORATORIO DE FÍSICA BÁSICA I}}\\
\vspace*{0.3cm}
{\Large \textbf{PRACTICA No. 2}}\\
\vspace*{3.5cm}
{\Large \textbf{MEDICIONES INDIRECTAS Y PROPAGACIÓN DE ERRORES}}\\
\end{center}

\vspace*{6.0cm}
\leftskip=7.95cm
\noindent
\textbf{Estudiantes:}\\
1. Caballero Burgoa, Carlos Eduardo.\\
2. Caballero Velarde, José Enrique.\\
3. Camacho Cuenca, Cinthya Paola.\\
\newline
\textbf{Docente:}\\
Ing. Oscar Garcia\\
\newline
\textbf{Horario del grupo:} Miércoles 15:45.\\
\textbf{Fecha de realización:} 7 de Mayo del 2014.\\
\textbf{Fecha de entrega:} 14 de Mayo del 2014.\\

\end{titlepage}

\blankpage

\section{Resumen}
En la practica anterior, se ha visto el procedimiento apropiado para realizar
mediciones directas para propiedades físicas del distintos objetos, además del
método para el calculo del error de dichas mediciones.

Pero además existen otras propiedades que no pueden ser medidas directamente con
instrumentos, para realizar mediciones sobre este tipo de propiedades se debe
utilizar una \emph{medición indirecta}, al momento de calcular el valor de una
medición indirecta, se deben tener en cuenta otros métodos para el calculo de su
error.

Teniendo en cuenta los datos calculados en la anterior experiencia, en este
informe se calcularán los valores de volumen ($V$) y densidad ($\rho$), para
el cilindro, disco, y esfera utilizados anteriormente.

\section{Objetivos}
\begin{itemize}
\item Realizar mediciones indirectas y comunicar correctamente los resultados.
\end{itemize}

\section{Fundamento teórico}
Las mediciones indirectas son mediciones donde no es posible obtener un valor
directamente con el instrumento de medición. Para determinar el valor de la
medición es necesaria una función matemática que relacione las magnitudes.

Para la determinación del error de las mediciones indirectas, se utiliza el
método de propagación de errores, es decir, la propagación o efecto que
producen los errores de las mediciones directas al error de la función.

\subsection{Propagación de errores}
Consideremos el calculo el valor de una magnitud $y$ que es función de una serie
de magnitudes $x_1, x_2, \dots, x_n$, cuyos valores se pueden obtener de una
manera directa en el laboratorio, es decir que:

\begin{equation}
\label{eq:f}
    y = f(x_1,x_2,\dots,x_n)
\end{equation}

donde $x_i$, son los resultados de mediciones directas, ellas son conocidas como
variables independientes:

\begin{equation}x_1 = (\bar{x}_1 \pm e_1)[u],\end{equation}
\begin{equation}x_2 = (\bar{x}_2 \pm e_2)[u],\end{equation}
\begin{equation}x_n = (\bar{x}_n \pm e_n)[u],\end{equation}

La mejor estimación de $y$ se obtiene sustituyendo en la expresión \ref{eq:f}
los valores obtenidos de $\bar{x}_i$:

\begin{equation}
    \bar{y} = f(\bar{x}_1,\bar{x}_2,\dots,\bar{x}_n)
\end{equation}

Para la estimación del error de $\bar{y}$ se utiliza la siguiente formula:

\begin{equation}
    e_y = \sqrt{\sum_{i=1}^{n}
    \left(\left|\frac{\partial{f}}{\partial{x_i}}\right|_{\bar{x}_i}
    e_i\right)^2}
\end{equation}

finalmente el resultado de la medición indirecta es:

\begin{equation}
    y = (\bar{y} \pm e_y)[u], E\%
\end{equation}

\subsection{Calculo del error del volumen de un cilindro}
Ahora procederemos a calcular la formula a utilizarse para el calculo del
error del volumen de un cilindro.

Dada la ecuación para el calculo del volumen de un cilindro:

\begin{equation}
    V = \frac{\pi D^2 H}{4}
\end{equation}

y conocidas las mediciones directas de $D$ y $H$:

\begin{equation}
    D = \bar{D} \pm e_D [cm]; E\%
\end{equation}
\begin{equation}
    H = \bar{H} \pm e_H [cm]; E\%
\end{equation}

Las derivadas parciales serian:

\begin{equation}
    \frac{\partial{V}}{\partial{D}} = \frac{\pi H D}{2}
\end{equation}
\begin{equation}
    \frac{\partial{V}}{\partial{H}} = \frac{\pi D^2}{4}
\end{equation}

Siendo el error de la medición:

\begin{equation}
    e_V = \sqrt{
        \left(\frac{\pi \bar{H} \bar{D}}{2}\right)^2 {e_D}^2 +
        \left(\frac{\pi \bar{D}^2}{4}\right)^2 {e_H}^2
    }
\end{equation}

\subsection{Calculo del error del volumen de una esfera}
Ahora procederemos a calcular la formula a utilizarse para el calculo del
error del volumen de una esfera.

Dada la ecuación para el calculo del volumen de una esfera:

\begin{equation}
    V = \frac{\pi D^3}{6}
\end{equation}

y conocido su diámetro $D$:

\begin{equation}
    D = \bar{D} \pm e_D [cm]; E\%
\end{equation}

La derivada parcial seria:

\begin{equation}
    \frac{\partial{V}}{\partial{D}} = \frac{\pi D^2}{2}
\end{equation}

Siendo el error de la medición:

\begin{equation}
    e_V = \frac{\pi \bar{D}^2}{2} e_D
\end{equation}

\subsection{Calculo del error de la densidad de un objeto}
Ahora procederemos a calcular la formula a utilizarse para el calculo del
error de la densidad de los objetos.

Dada la ecuación para el calculo de la densidad de un objeto:

\begin{equation}
    \rho = \frac{m}{V}
\end{equation}

y conocidas las mediciones de $m$ y $V$:

\begin{equation}
    m = \bar{m} \pm e_m [cm]; E\%
\end{equation}
\begin{equation}
    V = \bar{V} \pm e_V [cm^3]; E\%
\end{equation}

Las derivadas parciales serian:

\begin{equation}
    \frac{\partial{\rho}}{\partial{m}} = \frac{1}{V}
\end{equation}
\begin{equation}
    \frac{\partial{\rho}}{\partial{V}} = -\frac{m}{V^2}
\end{equation}

Siendo el error de la medición:

\begin{equation}
    e_{\rho} = \sqrt{
        \left(\frac{1}{\bar{V}}\right)^2 {e_m}^2 +
        \left(-\frac{\bar{m}}{\bar{V}^2}\right)^2 {e_V}^2
    }
\end{equation}

\section{Materiales y montaje experimental}
En esta práctica no se realizan mediciones. Sin embargo es necesario una
calculadora científica como herramienta de trabajo.

\section{Descripción del procedimiento experimental}
A continuación se describe el procedimiento experimental de calculo que se
llevará a cabo:

\begin{enumerate}
\item Copiar los resultados de las mediciones (valor representativo, su error y
su unidad) de la práctica anterior, es decir, los valores de diámetros, alturas,
y masas del cilindro, disco, y esfera utilizados.
\item Realizar las medidas indirectas del volumen y la densidad para el
cilindro, disco, y esfera.
\item Comunicar correctamente los resultados de las mediciones indirectas.
\end{enumerate}

\section{Registro de datos}

\subsection{Cilindro}
Medición de la longitud, diámetro, y masa del cilindro:

\vspace{0.4cm}
\begin{tabular}{|c|p{5.0cm}|}
\hline
\multicolumn{2}{|c|}{Resultados de la medición:} \\ \hline
$H$ & \\ \hline
$D$ & \\ \hline
$m$ & \\ \hline
\end{tabular}

\subsection{Disco}
Medición de la longitud, diámetro, y masa del disco:

\vspace{0.4cm}
\begin{tabular}{|c|p{5.0cm}|}
\hline
\multicolumn{2}{|c|}{Resultados de la medición:} \\ \hline
$H$ & \\ \hline
$D$ & \\ \hline
$m$ & \\ \hline
\end{tabular}

\subsection{Esfera}
Medición del diámetro y masa de la esfera:

\vspace{0.4cm}
\begin{tabular}{|c|p{5.0cm}|}
\hline
\multicolumn{2}{|c|}{Resultados de la medición:} \\ \hline
$D$ & \\ \hline
$m$ & \\ \hline
\end{tabular}

\section{Cálculos y tablas}

\subsection{Cilindro}
\newpage

\subsection{Disco}
\vspace{6.0cm}

\subsection{Esfera}
\vspace{6.0cm}

\subsection{Resumen de mediciones}
A continuación se resumen las medidas obtenidas:

\begin{center}
\begin{tabular}{|c|>{\centering}m{2.2cm}<{\centering}
                  |>{\centering}m{2.2cm}<{\centering}
                  |>{\centering}m{2.2cm}<{\centering}
                  |>{\centering}m{2.2cm}<{\centering}
                  |>{\centering}m{2.2cm}<{\centering}|}
\hline
\textbf{Objeto} & \textbf{$H[cm]$}
             & \textbf{$D[cm]$}
             & \textbf{$m[g]$}
             & \textbf{$V[cm^3]$}
             & \textbf{$\rho[g/cm^3]$} \tabularnewline \hline
\textbf{Cilindro} & & & & & \\ \hline
\textbf{Disco} & & & & & \\ \hline
\textbf{Esfera} & & & & & \\ \hline
\end{tabular}\\
\end{center}

\section{Conclusiones}

\section{Referencias bibliográficas}
\begin{itemize}
\item Errores de medición y su propagación \\
http://www.tplaboratorioquimico.com/2008/08/\
errores-de-medicion-y-su-propagacion.html
\item Tratamiento y propagación de errores \\
http://www.lawebdefisica.com/apuntsfis/errores/
\item Informe de propagación de errores \\
http://www.slideshare.net/cdloor/\
informe-de-propagacion-de-errores-laboratorio-de-fisica-c
\item Guía práctica para la realización de la medida y el calculo de errores \\
http://bacterio.uc3m.es/docencia/laboratorio/guiones\_esp/\
errores/guiondeerrores.pdf
\end{itemize}

\section{Respuestas al cuestionario}
\begin{enumerate}
\item ¿Que criterio utilizo para obtener el error del volumen y de la densidad
a partir de las contribuciones de los errores involucrados en cada una de ellas?
\item En la estimación del error del volumen de un cilindro se tiene la
contribución del error de su longitud y del error de su diámetro. ¿Cuál de ellos
contribuye más al error del volumen?
\item A partir del resultado de la pregunta anterior, la longitud o el diámetro
debería medirse con mayor precisión?
\item En la estimación del error del volumen de un disco se tiene la
contribución del error de su espesor (altura $H$) y de su diámetro. ¿Cuál de ellos contribuye mas al error del volumen?
\item A partir del resultado de la pregunta anterior, el espesor o el diámetro
debería medirse con mayor precisión?
\item En la estimación del error de la densidad se tiene la contribución del
error del volumen y de la masa. ¿Cuál de ellos contribuye más al error de la
densidad?
\item A partir del resultado de la pregunta anterior, la masa o el volumen
debería medirse con mayor precisión?
\item De la tabla resumen obtenga el valor de la densidad del cilindro, disco, y
esfera, compare estos valores con valores publicados en la literatura y diga
aproximadamente de que material están hechos.
\end{enumerate}

\end{document}
