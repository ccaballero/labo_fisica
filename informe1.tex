\documentclass[letter,twoside,11pt]{article}

\usepackage[spanish]{babel}
\usepackage[utf8]{inputenc}

\usepackage{lmodern}
\usepackage[T1]{fontenc}
\usepackage{textcomp}

\usepackage{graphicx}
\usepackage{pstricks}

\usepackage{anysize}
\marginsize{3cm}{2cm}{2cm}{3cm}

\usepackage{amsmath}

\usepackage{fancyhdr}
\usepackage{lastpage}
\pagestyle{fancy}
\fancyhf{}
\fancyhead[LE,RO]{Laboratorio de Física Básica I}
\fancyfoot[CO,CE]{\thepage\ de \pageref{LastPage}}

\special{papersize=215.9mm,279.4mm}

\newcommand{\blankpage}{
\newpage
\thispagestyle{empty}
\mbox{}
\newpage
}

\begin{document}

\begin{titlepage}
\begin{center}
{\Large UNIVERSIDAD MAYOR DE SAN SIMÓN}\\
\vspace*{0.15cm}
{\large FACULTAD DE CIENCIAS Y TECNOLOGÍA}\\
\vspace*{0.10cm}
DEPARTAMENTO DE FÍSICA\\
\vspace*{3.0cm}
{\Large \textbf{LABORATORIO DE FÍSICA BÁSICA I}}\\
\vspace*{0.3cm}
{\Large \textbf{PRACTICA No. 1}}\\
\vspace*{3.5cm}
{\Large \textbf{MEDICIONES DIRECTAS}}\\
\end{center}

\vspace*{6.0cm}
\leftskip=7.95cm
\noindent
\textbf{Estudiantes:}\\
1. Caballero Burgoa, Carlos Eduardo\\
2. Caballero Burgoa, Carlos Eduardo\\
\newline
\textbf{Docente:}\\
Ing. Oscar Garcia\\
\newline
\textbf{Horario del grupo:} Miercoles 15:45.\\
\textbf{Fecha de realización:} 9 de Abril del 2014.\\
\textbf{Fecha de entrega:} 30 de Abril del 2014.\\

\end{titlepage}

\blankpage

\section{Resumen}
\section{Objetivos}
\begin{itemize}
\item Medir diferentes magnitudes físicas: una medición y una serie de
mediciones.
\item Escribir los resultados de las mediciones.
\end{itemize}

\section{Fundamento teórico}
Los resultados de las medidas nunca se corresponden con los valores reales de
las magnitudes a medir, sino que, en mayor o menor extensión, son defectuosos,
es decir, están afectados de error. Las causas que motivan tales desviaciones
pueden ser debidas al observador, al aparato o incluso a las propias
características del proceso de medida.

Las mediciones directas son aquellos valores que se consiguen directamente con
la escala de un instrumento. Se pueden realizar en una sola medición o en una
serie de mediciones.

Si las fuentes de error son únicamente de carácter aleatorio, es decir, si
influyen unas veces por exceso y otras por defecto en el resultado de la medida,
puede demostrarse que el valor que más se aproxima al verdadero valor es
precisamente el valor medio. Ello es debido a que al promediar todos los
resultados, los errores por exceso tenderán a compensarse con los errores por
defecto y ello será tanto más cierto cuanto mayor sea el número de veces que se
repita la medida.

Si se realizan $n$ mediciones directas de una magnitud física, denotadas
por:
\begin{equation}
    \{x_1,x_2,x_3,\cdots,x_i,\cdots,x_n\}
\end{equation}

Para calcular el valor representativo de esta serie de mediciones, se toma la
media aritmetica:

\begin{equation}
    \bar{x} = \frac{x_1+x_2+\cdots+x_n}{n} = \frac{1}{n}\sum_{i=1}^{n} x_i
\end{equation}

Para determinar el error en la medición, se hace uso de la desviación estandar,
que es una medida de dispersión usada en estadística que nos dice cuánto tienden
a alejarse los valores concretos del promedio en una distribución.

La desviación estándar ($\sigma$) es la raíz cuadrada de la varianza ($s^2$) de
la distribución de probabilidad discreta, y puede calcularse con la siguiente
formula:

\begin{equation}
    \sigma = \sqrt{s^2}
\end{equation}
Donde:
\begin{equation}
    s^2 = \frac{1}{n}\sum_{i=1}^{n} (x_i-\bar{x})^2
\end{equation}

Aunque esta fórmula es correcta, en la práctica interesa el realizar inferencias
poblacionales, por lo que en el denominador en vez de $n$, se usa $n-1$ según la
corrección de Bessel.

\begin{equation}
    \sigma_{n-1} = \sqrt{\frac{\sum_{i=1}^{n} (x_i-\bar{x})^2}{(n-1)}}
\end{equation}

El error de la medida, es igual a la desviación estándar dividida por la raiz
cuadrada del número de mediciones.

\begin{equation}
    \sigma_x = \frac{\sigma_{n-1}}{\sqrt{n}}
             = \sqrt{\frac{\sum_{i=1}^{n} (x_i-\bar{x})}{n(n-1)}}
\end{equation}

Se define a $[(\bar{x}-\sigma_x), (\bar{x}+\sigma_x)]$ como el intervalo de
confianza en el que el valor verdadero de la medición puede encontrarse segun un
porcentaje de confianza definido por el modelo de distribucion de probabilidad.

Para el calculo del error $e_x$ de una serie de mediciones, es recomendable
colocar el mayor entre el error de la obtenido anteriormente $\sigma_x$ y la
precisión del instrumento de medida (P):

\begin{equation}
    e_x = \begin{cases}
        \sigma_x, & \mbox{si }\sigma_x > P_{ins} \\
        P,        & \mbox{si }\sigma_x < P_{ins}
    \end{cases}
\end{equation}

Finalmente el resultado de las mediciones será:

\begin{equation}
    x = (\bar{x}\pm e_x)[u], E\%
\end{equation}

\section{Materiales y montaje experimental}
\section{Descripción del procedimiento experimental}
\section{Registro de datos}
\section{Calculos y tablas}
\section{Conclusiones}
\section{Referencias biliográficas}
Fisicanet
http://fisicanet.com.ar/fisica/mediciones/ap01\_errores.php


https://es.wikipedia.org/wiki/Desviaci%C3%B3n\_t%C3%ADpica

\section{Respuestas al cuestionario}

\end{document}

