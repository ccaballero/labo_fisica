\documentclass[letter,twoside,11pt]{article}

\usepackage[spanish]{babel}
\usepackage[utf8]{inputenc}

\usepackage{lmodern}
\usepackage[T1]{fontenc}
\usepackage{textcomp}

\usepackage{graphicx}
\usepackage{pstricks}

\usepackage{anysize}
\marginsize{3cm}{2cm}{2cm}{3cm}

\usepackage{amsmath}
\usepackage{array}

\usepackage{fancyhdr}
\usepackage{lastpage}
\pagestyle{fancy}
\fancyhf{}
\fancyhead[LE,RO]{Laboratorio de Física Básica I}
\fancyfoot[CO,CE]{\thepage\ de \pageref{LastPage}}

\special{papersize=215.9mm,279.4mm}

\newcommand{\blankpage}{
\newpage
\thispagestyle{empty}
\mbox{}
\newpage
}

\renewcommand{\arraystretch}{1.2}

\begin{document}

\begin{titlepage}
\begin{center}
{\Large UNIVERSIDAD MAYOR DE SAN SIMÓN}\\
\vspace*{0.15cm}
{\large FACULTAD DE CIENCIAS Y TECNOLOGÍA}\\
\vspace*{0.10cm}
DEPARTAMENTO DE FÍSICA\\
\vspace*{3.0cm}
{\Large \textbf{LABORATORIO DE FÍSICA BÁSICA I}}\\
\vspace*{0.3cm}
{\Large \textbf{PRACTICA No. 1}}\\
\vspace*{3.5cm}
{\Large \textbf{MEDICIONES DIRECTAS}}\\
\end{center}

\vspace*{6.0cm}
\leftskip=7.95cm
\noindent
\textbf{Estudiantes:}\\
1. \\
2. \\
\newline
\textbf{Docente:}\\
Ing. Oscar Garcia\\
\newline
\textbf{Horario del grupo:} Miércoles 15:45.\\
\textbf{Fecha de realización:} 9 de Abril del 2014.\\
\textbf{Fecha de entrega:} 30 de Abril del 2014.\\

\end{titlepage}

\blankpage

\section{Resumen}
El presente informe resume la experiencia de laboratorio en el calculo de
mediciones directas, para esto se han calculado las dimensiones de los
diferentes objetos provistos, ademas de realizar una medición de tiempo con la
ayuda de un péndulo.

Para realizar este calculo de medidas, se han utilizado un conjunto de
herramientas propias de la estadística descriptiva. Con estas herramientas se
han conseguido las mediciones directas, ademas de obtener los intervalos de
confianza para tales mediciones.

\section{Objetivos}
\begin{itemize}
\item Medir diferentes magnitudes físicas: una medición y una serie de
mediciones.
\item Escribir los resultados de las mediciones.
\end{itemize}

\section{Fundamento teórico}
Los resultados de las medidas nunca se corresponden con los valores reales de
las magnitudes a medir, sino que, en mayor o menor extensión, son defectuosos,
es decir, están afectados de error. Las causas que motivan tales desviaciones
pueden ser debidas al observador, al aparato o incluso a las propias
características del proceso de medida.

Las mediciones directas son aquellos valores que se consiguen directamente con
la escala de un instrumento. Se pueden realizar en una sola medición o en una
serie de mediciones.

Si las fuentes de error son únicamente de carácter aleatorio, es decir, si
influyen unas veces por exceso y otras por defecto en el resultado de la medida,
puede demostrarse que el valor que más se aproxima al verdadero valor es
precisamente el valor medio. Ello es debido a que al promediar todos los
resultados, los errores por exceso tenderán a compensarse con los errores por
defecto y ello será tanto más cierto cuanto mayor sea el número de veces que se
repita la medida.

Si se realizan $n$ mediciones directas de una magnitud física, denotadas
por:
\begin{equation}
    \{x_1,x_2,x_3,\cdots,x_i,\cdots,x_n\}
\end{equation}

Para calcular el valor representativo de esta serie de mediciones, se toma la
media aritmética:

\begin{equation}
    \bar{x} = \frac{x_1+x_2+\cdots+x_n}{n} = \frac{1}{n}\sum_{i=1}^{n} x_i
\end{equation}

Para determinar el error en la medición, se hace uso de la desviación estándar,
que es una medida de dispersión usada en estadística que nos dice cuánto tienden
a alejarse los valores concretos del promedio en una distribución.

La desviación estándar ($\sigma$) es la raíz cuadrada de la varianza ($s^2$) de
la distribución de probabilidad discreta, y puede calcularse con la siguiente
formula:

\begin{equation}
    \sigma = \sqrt{s^2}
\end{equation}
Donde:
\begin{equation}
    s^2 = \frac{1}{n}\sum_{i=1}^{n} (x_i-\bar{x})^2
\end{equation}

Aunque esta fórmula es correcta, en la práctica interesa el realizar inferencias
poblacionales, por lo que en el denominador en vez de $n$, se usa $n-1$ según la
corrección de Bessel.

\begin{equation}
    \sigma_{n-1} = \sqrt{\frac{\sum_{i=1}^{n} (x_i-\bar{x})^2}{(n-1)}}
\end{equation}

El error de la medida, es igual a la desviación estándar dividida por la raíz
cuadrada del número de mediciones.

\begin{equation}
    \sigma_x = \frac{\sigma_{n-1}}{\sqrt{n}}
             = \sqrt{\frac{\sum_{i=1}^{n} (x_i-\bar{x})}{n(n-1)}}
\end{equation}

Se define a $[(\bar{x}-\sigma_x), (\bar{x}+\sigma_x)]$ como el intervalo de
confianza en el que el valor verdadero de la medición puede encontrarse según un
porcentaje de confianza definido por el modelo de distribución de probabilidad.

Para el calculo del error $e_x$ de una serie de mediciones, es recomendable
colocar el mayor entre el error de la obtenido anteriormente $\sigma_x$ y la
precisión del instrumento de medida (P):

\begin{equation}
    e_x = \begin{cases}
        \sigma_x, & \mbox{si }\sigma_x > P_{ins} \\
        P,        & \mbox{si }\sigma_x < P_{ins}
    \end{cases}
\end{equation}

Finalmente el resultado de las mediciones será:

\begin{equation}
    x = (\bar{x}\pm e_x)[u], E\%
\end{equation}

\section{Materiales y montaje experimental}
\begin{itemize}
\item Cilindro metálico.
\item Disco metálico.
\item Esfera metálica.
\item Calibrador con precisión 0.2mm.
\item Tornillo micrómetro con precisión 0.1mm.
\item Balanza con precisión de 0.01g.
\item Cronometro con precisión de 0.01s.
\item Péndulo.
\item Regla milimétrica.
\end{itemize}

\section{Descripción del procedimiento experimental}
A continuación se describe el procedimiento experimental de medición que se
llevará a cabo.

\subsection{Péndulo}
\begin{enumerate}
\item Armar el equipo como se muestra en la figura \ref{péndulo}. Si es
necesario, nivelar el equipo.
\item Fijar la longitud para el péndulo y medir esa longitud.
\item Registrar los valores del tiempo de diez oscilaciones para un angulo menor
o igual a diez grados.
\item Calcular el error de la media aritmética.
\item Escribir los resultados de los cálculos, para el valor medio ($\bar{t}$),
el error de la media aritmética ($\sigma_t$), la precisión del instrumento ($P$)
y el error de la medición ($e_t$). Finalmente escribir los resultados de la
medición de la longitud y el tiempo de 10 oscilaciones.
\end{enumerate}

\begin{figure}
\centering
%LaTeX with PSTricks extensions
%%Creator: inkscape 0.48.4
%%Please note this file requires PSTricks extensions
\psset{xunit=.5pt,yunit=.5pt,runit=.5pt}
\begin{pspicture}(177.16534424,301.18109131)
{
\newrgbcolor{curcolor}{0 0 0}
\pscustom[linewidth=0.6185259,linecolor=curcolor]
{
\newpath
\moveto(81.29573757,250.10657437)
\lineto(81.29573757,126.96296538)
}
}
{
\newrgbcolor{curcolor}{0 0 0}
\pscustom[linewidth=0.61852589,linecolor=curcolor]
{
\newpath
\moveto(88.31173697,119.9864439)
\curveto(88.31173697,116.11161378)(85.17056651,112.97044332)(81.29573638,112.97044332)
\curveto(77.42090626,112.97044332)(74.2797358,116.11161378)(74.2797358,119.9864439)
\curveto(74.2797358,123.86127403)(77.42090626,127.00244449)(81.29573638,127.00244449)
\curveto(85.17056651,127.00244449)(88.31173697,123.86127403)(88.31173697,119.9864439)
\closepath
}
}
{
\newrgbcolor{curcolor}{0 0 0}
\pscustom[linewidth=0.6185259,linecolor=curcolor]
{
\newpath
\moveto(43.5641078,11.84351545)
\lineto(9.10427821,82.50322674)
\lineto(13.46338977,86.85917888)
\lineto(47.57929781,26.07240272)
\closepath
}
}
{
\newrgbcolor{curcolor}{0 0 0}
\pscustom[linewidth=0.6185259,linecolor=curcolor]
{
\newpath
\moveto(163.060651,26.07240272)
\lineto(81.81177723,86.85917888)
\lineto(13.46338977,86.85917888)
\lineto(47.57929781,26.07240272)
\closepath
}
}
{
\newrgbcolor{curcolor}{0 0 0}
\pscustom[linewidth=0.6185259,linecolor=curcolor]
{
\newpath
\moveto(171.48067437,11.84351545)
\lineto(43.5641078,11.84351545)
\lineto(47.57929781,26.07240272)
\lineto(163.060651,26.07240272)
\closepath
}
}
{
\newrgbcolor{curcolor}{0 0 0}
\pscustom[linewidth=0.6185259,linecolor=curcolor]
{
\newpath
\moveto(171.48067437,11.84351545)
\lineto(81.62526375,82.50322674)
\lineto(81.81177723,86.85917888)
\lineto(163.060651,26.07240272)
\closepath
}
}
{
\newrgbcolor{curcolor}{0 0 0}
\pscustom[linewidth=0.6185259,linecolor=curcolor]
{
\newpath
\moveto(171.48067437,11.84351545)
\lineto(81.62526375,82.50322674)
\lineto(9.10427821,82.50322674)
\lineto(43.5641078,11.84351545)
\closepath
}
}
{
\newrgbcolor{curcolor}{0 0 0}
\pscustom[linewidth=0.6185259,linecolor=curcolor]
{
\newpath
\moveto(81.62526375,82.50322674)
\lineto(9.10427821,82.50322674)
\lineto(13.46338977,86.85917888)
\lineto(81.81177723,86.85917888)
\closepath
}
}
{
\newrgbcolor{curcolor}{1 1 1}
\pscustom[linestyle=none,fillstyle=solid,fillcolor=curcolor]
{
\newpath
\moveto(15.6443892,83.59637162)
\lineto(17.14395318,80.94266644)
\lineto(80.63005594,81.11199134)
\lineto(80.19312439,83.65781736)
\closepath
}
}
{
\newrgbcolor{curcolor}{0 0 0}
\pscustom[linewidth=0.6185259,linecolor=curcolor]
{
\newpath
\moveto(55.37021137,68.46001754)
\lineto(55.37021137,284.0283999)
}
}
{
\newrgbcolor{curcolor}{0 0 0}
\pscustom[linewidth=0.6185259,linecolor=curcolor]
{
\newpath
\moveto(59.33772589,68.46001754)
\lineto(59.33772589,284.0283999)
}
}
{
\newrgbcolor{curcolor}{0 0 0}
\pscustom[linewidth=0.6185259,linecolor=curcolor]
{
\newpath
\moveto(55.378325,68.44875831)
\curveto(56.33945502,67.24084337)(58.52796884,67.27980555)(59.31375681,68.44875831)
}
}
{
\newrgbcolor{curcolor}{1 1 1}
\pscustom[linestyle=none,fillstyle=solid,fillcolor=curcolor]
{
\newpath
\moveto(55.66243274,89.14155551)
\lineto(59.02140809,89.14155551)
\lineto(59.02140809,79.79167285)
\lineto(55.66243274,79.79167285)
\closepath
}
}
{
\newrgbcolor{curcolor}{1 1 1}
\pscustom[linestyle=none,fillstyle=solid,fillcolor=curcolor]
{
\newpath
\moveto(142.82117466,25.75445944)
\lineto(162.41561877,25.75445944)
\curveto(162.68990788,25.75445944)(163.02328862,25.5644859)(163.16310922,25.3285098)
\lineto(170.72499054,12.56627478)
\curveto(170.86481114,12.33029868)(170.75655669,12.14032514)(170.48226758,12.14032514)
\lineto(150.88782347,12.14032514)
\curveto(150.61353436,12.14032514)(150.28015362,12.33029868)(150.14033302,12.56627478)
\lineto(142.5784517,25.3285098)
\curveto(142.4386311,25.5644859)(142.54688555,25.75445944)(142.82117466,25.75445944)
\closepath
}
}
{
\newrgbcolor{curcolor}{1 1 1}
\pscustom[linestyle=none,fillstyle=solid,fillcolor=curcolor]
{
\newpath
\moveto(9.48204381,82.44281521)
\lineto(13.38735814,86.3552328)
\lineto(16.83151907,80.23169389)
\lineto(10.6809793,79.98997824)
\closepath
}
}
{
\newrgbcolor{curcolor}{1 1 1}
\pscustom[linestyle=none,fillstyle=solid,fillcolor=curcolor]
{
\newpath
\moveto(80.45647697,86.51228601)
\lineto(81.76796115,86.5894423)
\lineto(82.96372659,85.58655013)
\lineto(82.96372659,79.10625938)
\lineto(80.14789195,79.10625938)
\closepath
}
}
{
\newrgbcolor{curcolor}{1 1 1}
\pscustom[linestyle=none,fillstyle=solid,fillcolor=curcolor]
{
\newpath
\moveto(151.02686287,26.38930912)
\lineto(154.87267643,26.37534418)
\lineto(82.32727096,83.88897203)
\lineto(82.01868593,77.56299386)
\closepath
}
}
{
\newrgbcolor{curcolor}{0 0 0}
\pscustom[linewidth=0.6185259,linecolor=curcolor]
{
\newpath
\moveto(74.08225681,257.3348164)
\lineto(87.31475567,257.3348164)
\curveto(87.36764243,257.3348164)(87.41021913,257.2922397)(87.41021913,257.23935295)
\lineto(87.41021913,250.71856698)
\curveto(87.41021913,250.66568022)(87.36764243,250.62310352)(87.31475567,250.62310352)
\lineto(74.08225681,250.62310352)
\curveto(74.02937006,250.62310352)(73.98679336,250.66568022)(73.98679336,250.71856698)
\lineto(73.98679336,257.23935295)
\curveto(73.98679336,257.2922397)(74.02937006,257.3348164)(74.08225681,257.3348164)
\closepath
}
}
{
\newrgbcolor{curcolor}{0 0 0}
\pscustom[linewidth=0.6185259,linecolor=curcolor]
{
\newpath
\moveto(50.89071308,290.00175083)
\lineto(58.69719968,290.00175083)
\curveto(58.75786487,290.00175083)(58.80670363,289.95291207)(58.80670363,289.89224688)
\lineto(58.80670363,285.8463977)
\curveto(58.80670363,285.78573251)(58.75786487,285.73689375)(58.69719968,285.73689375)
\lineto(50.89071308,285.73689375)
\curveto(50.83004789,285.73689375)(50.78120913,285.78573251)(50.78120913,285.8463977)
\lineto(50.78120913,289.89224688)
\curveto(50.78120913,289.95291207)(50.83004789,290.00175083)(50.89071308,290.00175083)
\closepath
}
}
{
\newrgbcolor{curcolor}{0 0 0}
\pscustom[linewidth=0.6185259,linecolor=curcolor]
{
\newpath
\moveto(58.77443991,290.00906449)
\lineto(87.41952733,257.36387013)
}
}
{
\newrgbcolor{curcolor}{0 0 0}
\pscustom[linewidth=0.6185259,linecolor=curcolor]
{
\newpath
\moveto(50.82432525,289.98775923)
\lineto(74.01934172,257.25938619)
}
}
{
\newrgbcolor{curcolor}{0 0 0}
\pscustom[linewidth=0.6185259,linecolor=curcolor]
{
\newpath
\moveto(50.85628052,285.73201359)
\lineto(73.94219416,250.81694042)
}
}
{
\newrgbcolor{curcolor}{1 1 1}
\pscustom[linestyle=none,fillstyle=solid,fillcolor=curcolor]
{
\newpath
\moveto(51.40056371,289.7108816)
\lineto(58.68699634,289.72449741)
\lineto(69.28756758,277.54904752)
\lineto(60.15237066,277.35812235)
\closepath
}
}
{
\newrgbcolor{curcolor}{1 1 1}
\pscustom[linestyle=none,fillstyle=solid,fillcolor=curcolor]
{
\newpath
\moveto(51.08344262,289.0774581)
\lineto(51.07663471,285.96125855)
\lineto(60.78663205,271.27352293)
\lineto(60.78663205,275.41254611)
\closepath
}
}
\end{pspicture}

\caption{Péndulo simple}
\label{pendulo}
\end{figure}

\subsection{Cilindro}
\begin{enumerate}
\item Con el \emph{calibrador}, medir seis veces la altura $H$ del cilindro, y
con el \emph{tornillo micrómetro} medir seis veces su diámetro $D$ (figura
\ref{objetos}).
\item Con la balanza medir una sola vez la masa $m$ del cilindro.
\item Determinar los valores representativos, los errores y escribir el
resultado de la medición para cada una de ellas.
\end{enumerate}

\subsection{Disco}
\begin{enumerate}
\item Con el \emph{calibrador}, medir seis veces el diámetro $D$ del disco, y
con el \emph{tornillo micrómetro} medir seis veces su altura $H$ (figura
\ref{objetos}).
\item Con la balanza medir una sola vez la masa $m$ del disco.
\item Determinar los valores representativos, los errores y escribir el
resultado de la medición para cada una de ellas.
\end{enumerate}

\subsection{Esfera}
\begin{enumerate}
\item Con el \emph{tornillo micrómetro}, medir seis veces el diámetro $D$ de
la esfera (figura \ref{objetos}).
\item Con la balanza medir una sola vez la masa $m$ de la esfera.
\item Determinar los valores representativos, los errores y escribir el
resultado de la medición para cada una de ellas.
\end{enumerate}

\begin{figure}
\centering
%LaTeX with PSTricks extensions
%%Creator: inkscape 0.48.4
%%Please note this file requires PSTricks extensions
\psset{xunit=.5pt,yunit=.5pt,runit=.5pt}
\begin{pspicture}(708.66137695,248.03149414)
{
\newrgbcolor{curcolor}{1 1 1}
\pscustom[linestyle=none,fillstyle=solid,fillcolor=curcolor]
{
\newpath
\moveto(188.93517488,190.92307848)
\curveto(188.93517488,184.67064237)(163.59216985,179.60204144)(132.32998883,179.60204144)
\curveto(101.06780782,179.60204144)(75.72480278,184.67064237)(75.72480278,190.92307848)
\curveto(75.72480278,197.17551459)(101.06780782,202.24411552)(132.32998883,202.24411552)
\curveto(163.59216985,202.24411552)(188.93517488,197.17551459)(188.93517488,190.92307848)
\closepath
}
}
{
\newrgbcolor{curcolor}{0 0 0}
\pscustom[linewidth=0.70866141,linecolor=curcolor]
{
\newpath
\moveto(188.93517488,190.92307848)
\curveto(188.93517488,184.67064237)(163.59216985,179.60204144)(132.32998883,179.60204144)
\curveto(101.06780782,179.60204144)(75.72480278,184.67064237)(75.72480278,190.92307848)
\curveto(75.72480278,197.17551459)(101.06780782,202.24411552)(132.32998883,202.24411552)
\curveto(163.59216985,202.24411552)(188.93517488,197.17551459)(188.93517488,190.92307848)
\closepath
}
}
{
\newrgbcolor{curcolor}{1 1 1}
\pscustom[linestyle=none,fillstyle=solid,fillcolor=curcolor]
{
\newpath
\moveto(188.93517488,57.10841493)
\curveto(188.93517488,50.85597882)(163.59216985,45.78737789)(132.32998883,45.78737789)
\curveto(101.06780782,45.78737789)(75.72480278,50.85597882)(75.72480278,57.10841493)
\curveto(75.72480278,63.36085104)(101.06780782,68.42945197)(132.32998883,68.42945197)
\curveto(163.59216985,68.42945197)(188.93517488,63.36085104)(188.93517488,57.10841493)
\closepath
}
}
{
\newrgbcolor{curcolor}{0 0 0}
\pscustom[linewidth=0.70866141,linecolor=curcolor]
{
\newpath
\moveto(188.93517488,57.10841493)
\curveto(188.93517488,50.85597882)(163.59216985,45.78737789)(132.32998883,45.78737789)
\curveto(101.06780782,45.78737789)(75.72480278,50.85597882)(75.72480278,57.10841493)
\curveto(75.72480278,63.36085104)(101.06780782,68.42945197)(132.32998883,68.42945197)
\curveto(163.59216985,68.42945197)(188.93517488,63.36085104)(188.93517488,57.10841493)
\closepath
}
}
{
\newrgbcolor{curcolor}{0 0 0}
\pscustom[linewidth=0.70866141,linecolor=curcolor]
{
\newpath
\moveto(75.72480287,190.54935209)
\lineto(75.72480287,57.6633755)
}
}
{
\newrgbcolor{curcolor}{0 0 0}
\pscustom[linewidth=0.70866141,linecolor=curcolor]
{
\newpath
\moveto(188.93517151,190.54935209)
\lineto(188.93517151,57.6633755)
}
}
{
\newrgbcolor{curcolor}{1 1 1}
\pscustom[linestyle=none,fillstyle=solid,fillcolor=curcolor]
{
\newpath
\moveto(188.57803318,58.65264674)
\curveto(188.57803318,52.27693259)(163.39492629,47.10839507)(132.32998963,47.10839507)
\curveto(101.26505296,47.10839507)(76.08194607,52.27693259)(76.08194607,58.65264674)
\curveto(76.08194607,65.02836088)(101.26505296,70.1968984)(132.32998963,70.1968984)
\curveto(163.39492629,70.1968984)(188.57803318,65.02836088)(188.57803318,58.65264674)
\closepath
}
}
{
\newrgbcolor{curcolor}{0 0 0}
\pscustom[linewidth=0.70866144,linecolor=curcolor]
{
\newpath
\moveto(57.560299,191.16874528)
\lineto(46.106797,191.16874528)
\lineto(46.106797,56.33657828)
\lineto(57.471013,56.33657828)
}
}
{
\newrgbcolor{curcolor}{1 1 1}
\pscustom[linestyle=none,fillstyle=solid,fillcolor=curcolor]
{
\newpath
\moveto(33.54262064,134.6028304)
\lineto(58.67097187,134.6028304)
\curveto(58.70572472,134.6028304)(58.73370265,134.57485247)(58.73370265,134.54009962)
\lineto(58.73370265,112.94728229)
\curveto(58.73370265,112.91252944)(58.70572472,112.88455151)(58.67097187,112.88455151)
\lineto(33.54262064,112.88455151)
\curveto(33.50786779,112.88455151)(33.47988986,112.91252944)(33.47988986,112.94728229)
\lineto(33.47988986,134.54009962)
\curveto(33.47988986,134.57485247)(33.50786779,134.6028304)(33.54262064,134.6028304)
\closepath
}
}
{
\newrgbcolor{curcolor}{0 0 0}
\pscustom[linestyle=none,fillstyle=solid,fillcolor=curcolor]
{
\newpath
\moveto(40.43008533,116.58260865)
\lineto(42.33569504,116.58260865)
\lineto(42.33569504,123.14191787)
\lineto(49.87789769,123.14191787)
\lineto(49.87789769,116.58260865)
\lineto(51.78350741,116.58260865)
\lineto(51.78350741,130.90477007)
\lineto(49.87789769,130.90477007)
\lineto(49.87789769,124.74664184)
\lineto(42.33569504,124.74664184)
\lineto(42.33569504,130.90477007)
\lineto(40.43008533,130.90477007)
\lineto(40.43008533,116.58260865)
}
}
{
\newrgbcolor{curcolor}{0 0 0}
\pscustom[linewidth=0.70866144,linecolor=curcolor]
{
\newpath
\moveto(75.401221,211.87950028)
\lineto(75.274952,225.26402228)
\lineto(188.285766,225.26402228)
\lineto(188.285766,211.87950128)
}
}
{
\newrgbcolor{curcolor}{1 1 1}
\pscustom[linestyle=none,fillstyle=solid,fillcolor=curcolor]
{
\newpath
\moveto(118.90049944,235.3655976)
\lineto(144.02885066,235.3655976)
\curveto(144.06360352,235.3655976)(144.09158145,235.33761967)(144.09158145,235.30286681)
\lineto(144.09158145,213.71004949)
\curveto(144.09158145,213.67529664)(144.06360352,213.64731871)(144.02885066,213.64731871)
\lineto(118.90049944,213.64731871)
\curveto(118.86574658,213.64731871)(118.83776866,213.67529664)(118.83776866,213.71004949)
\lineto(118.83776866,235.30286681)
\curveto(118.83776866,235.33761967)(118.86574658,235.3655976)(118.90049944,235.3655976)
\closepath
}
}
{
\newrgbcolor{curcolor}{0 0 0}
\pscustom[linestyle=none,fillstyle=solid,fillcolor=curcolor]
{
\newpath
\moveto(125.56732204,217.34539496)
\lineto(130.48178919,217.34539496)
\curveto(135.25583822,217.34539496)(137.3620432,220.05337145)(137.3620432,224.84747951)
\curveto(137.3620432,229.46105631)(134.89477569,231.66755638)(130.48178919,231.66755638)
\lineto(125.56732204,231.66755638)
\lineto(125.56732204,217.34539496)
\moveto(127.47293175,230.06283241)
\lineto(130.68237968,230.06283241)
\curveto(134.19270986,230.06283241)(135.45643349,227.99674689)(135.45643349,224.58671187)
\curveto(135.45643349,219.3111871)(131.94609851,218.95011893)(130.64226159,218.95011893)
\lineto(127.47293175,218.95011893)
\lineto(127.47293175,230.06283241)
}
}
{
\newrgbcolor{curcolor}{0 0 0}
\pscustom[linewidth=0.70866144,linecolor=curcolor]
{
\newpath
\moveto(665.57413668,124.01574441)
\curveto(665.57413668,86.7764645)(635.38573562,56.58806344)(598.14645571,56.58806344)
\curveto(560.90717579,56.58806344)(530.71877474,86.7764645)(530.71877474,124.01574441)
\curveto(530.71877474,161.25502433)(560.90717579,191.44342538)(598.14645571,191.44342538)
\curveto(635.38573562,191.44342538)(665.57413668,161.25502433)(665.57413668,124.01574441)
\closepath
}
}
{
\newrgbcolor{curcolor}{0 0 0}
\pscustom[linewidth=0.70866144,linecolor=curcolor]
{
\newpath
\moveto(525.984296,199.90262828)
\lineto(526.036596,213.28715028)
\lineto(662.261697,213.28715028)
\lineto(662.440267,199.18834328)
}
}
{
\newrgbcolor{curcolor}{1 1 1}
\pscustom[linestyle=none,fillstyle=solid,fillcolor=curcolor]
{
\newpath
\moveto(582.69786044,223.3887256)
\lineto(607.82621166,223.3887256)
\curveto(607.86096452,223.3887256)(607.88894245,223.36074767)(607.88894245,223.32599481)
\lineto(607.88894245,201.73317749)
\curveto(607.88894245,201.69842464)(607.86096452,201.67044671)(607.82621166,201.67044671)
\lineto(582.69786044,201.67044671)
\curveto(582.66310758,201.67044671)(582.63512966,201.69842464)(582.63512966,201.73317749)
\lineto(582.63512966,223.32599481)
\curveto(582.63512966,223.36074767)(582.66310758,223.3887256)(582.69786044,223.3887256)
\closepath
}
}
{
\newrgbcolor{curcolor}{0 0 0}
\pscustom[linestyle=none,fillstyle=solid,fillcolor=curcolor]
{
\newpath
\moveto(589.36468304,205.36852296)
\lineto(594.27915019,205.36852296)
\curveto(599.05319922,205.36852296)(601.1594042,208.07649945)(601.1594042,212.87060751)
\curveto(601.1594042,217.48418431)(598.69213669,219.69068438)(594.27915019,219.69068438)
\lineto(589.36468304,219.69068438)
\lineto(589.36468304,205.36852296)
\moveto(591.27029275,218.08596041)
\lineto(594.47974068,218.08596041)
\curveto(597.99007086,218.08596041)(599.25379449,216.01987489)(599.25379449,212.60983987)
\curveto(599.25379449,207.3343151)(595.74345951,206.97324693)(594.43962259,206.97324693)
\lineto(591.27029275,206.97324693)
\lineto(591.27029275,218.08596041)
}
}
{
\newrgbcolor{curcolor}{1 1 1}
\pscustom[linestyle=none,fillstyle=solid,fillcolor=curcolor]
{
\newpath
\moveto(460.36232849,119.72260232)
\curveto(460.36232849,95.17536083)(421.24177728,75.27588982)(372.98413177,75.27588982)
\curveto(324.72648626,75.27588982)(285.60593505,95.17536083)(285.60593505,119.72260232)
\curveto(285.60593505,144.26984381)(324.72648626,164.16931481)(372.98413177,164.16931481)
\curveto(421.24177728,164.16931481)(460.36232849,144.26984381)(460.36232849,119.72260232)
\closepath
}
}
{
\newrgbcolor{curcolor}{0 0 0}
\pscustom[linewidth=0.70866144,linecolor=curcolor]
{
\newpath
\moveto(460.36232849,119.72260232)
\curveto(460.36232849,95.17536083)(421.24177728,75.27588982)(372.98413177,75.27588982)
\curveto(324.72648626,75.27588982)(285.60593505,95.17536083)(285.60593505,119.72260232)
\curveto(285.60593505,144.26984381)(324.72648626,164.16931481)(372.98413177,164.16931481)
\curveto(421.24177728,164.16931481)(460.36232849,144.26984381)(460.36232849,119.72260232)
\closepath
}
}
{
\newrgbcolor{curcolor}{1 1 1}
\pscustom[linestyle=none,fillstyle=solid,fillcolor=curcolor]
{
\newpath
\moveto(460.36232849,128.30889232)
\curveto(460.36232849,103.76165083)(421.24177728,83.86217982)(372.98413177,83.86217982)
\curveto(324.72648626,83.86217982)(285.60593505,103.76165083)(285.60593505,128.30889232)
\curveto(285.60593505,152.85613381)(324.72648626,172.75560481)(372.98413177,172.75560481)
\curveto(421.24177728,172.75560481)(460.36232849,152.85613381)(460.36232849,128.30889232)
\closepath
}
}
{
\newrgbcolor{curcolor}{0 0 0}
\pscustom[linewidth=0.70866144,linecolor=curcolor]
{
\newpath
\moveto(460.36232849,128.30889232)
\curveto(460.36232849,103.76165083)(421.24177728,83.86217982)(372.98413177,83.86217982)
\curveto(324.72648626,83.86217982)(285.60593505,103.76165083)(285.60593505,128.30889232)
\curveto(285.60593505,152.85613381)(324.72648626,172.75560481)(372.98413177,172.75560481)
\curveto(421.24177728,172.75560481)(460.36232849,152.85613381)(460.36232849,128.30889232)
\closepath
}
}
{
\newrgbcolor{curcolor}{0 0 0}
\pscustom[linewidth=0.70866144,linecolor=curcolor]
{
\newpath
\moveto(285.613608,118.52435928)
\lineto(285.613608,128.37388928)
}
}
{
\newrgbcolor{curcolor}{0 0 0}
\pscustom[linewidth=0.70866144,linecolor=curcolor]
{
\newpath
\moveto(460.330768,118.52435928)
\lineto(460.330768,128.37389928)
}
}
{
\newrgbcolor{curcolor}{0 0 0}
\pscustom[linewidth=0.70866144,linecolor=curcolor]
{
\newpath
\moveto(284.252093,182.65298828)
\lineto(284.125823,198.71607828)
\lineto(459.993783,198.71607828)
\lineto(459.993783,182.47441828)
}
}
{
\newrgbcolor{curcolor}{1 1 1}
\pscustom[linestyle=none,fillstyle=solid,fillcolor=curcolor]
{
\newpath
\moveto(359.71565269,208.8176577)
\lineto(384.84400391,208.8176577)
\curveto(384.87875677,208.8176577)(384.9067347,208.78967977)(384.9067347,208.75492691)
\lineto(384.9067347,187.16210959)
\curveto(384.9067347,187.12735674)(384.87875677,187.09937881)(384.84400391,187.09937881)
\lineto(359.71565269,187.09937881)
\curveto(359.68089983,187.09937881)(359.65292191,187.12735674)(359.65292191,187.16210959)
\lineto(359.65292191,208.75492691)
\curveto(359.65292191,208.78967977)(359.68089983,208.8176577)(359.71565269,208.8176577)
\closepath
}
}
{
\newrgbcolor{curcolor}{0 0 0}
\pscustom[linestyle=none,fillstyle=solid,fillcolor=curcolor]
{
\newpath
\moveto(366.38247529,190.79745506)
\lineto(371.29694244,190.79745506)
\curveto(376.07099147,190.79745506)(378.17719645,193.50543155)(378.17719645,198.29953961)
\curveto(378.17719645,202.91311641)(375.70992894,205.11961648)(371.29694244,205.11961648)
\lineto(366.38247529,205.11961648)
\lineto(366.38247529,190.79745506)
\moveto(368.288085,203.51489251)
\lineto(371.49753293,203.51489251)
\curveto(375.00786311,203.51489251)(376.27158674,201.44880699)(376.27158674,198.03877197)
\curveto(376.27158674,192.7632472)(372.76125176,192.40217903)(371.45741484,192.40217903)
\lineto(368.288085,192.40217903)
\lineto(368.288085,203.51489251)
}
}
{
\newrgbcolor{curcolor}{0 0 0}
\pscustom[linewidth=0.70866144,linecolor=curcolor]
{
\newpath
\moveto(272.747503,131.21143828)
\lineto(261.472573,131.21143828)
\lineto(261.294003,119.41503828)
\lineto(273.015363,119.23646828)
}
}
{
\newrgbcolor{curcolor}{1 1 1}
\pscustom[linestyle=none,fillstyle=solid,fillcolor=curcolor]
{
\newpath
\moveto(233.55126064,135.89552162)
\lineto(258.67961187,135.89552162)
\curveto(258.71436472,135.89552162)(258.74234265,135.86754369)(258.74234265,135.83279084)
\lineto(258.74234265,114.23997351)
\curveto(258.74234265,114.20522066)(258.71436472,114.17724273)(258.67961187,114.17724273)
\lineto(233.55126064,114.17724273)
\curveto(233.51650779,114.17724273)(233.48852986,114.20522066)(233.48852986,114.23997351)
\lineto(233.48852986,135.83279084)
\curveto(233.48852986,135.86754369)(233.51650779,135.89552162)(233.55126064,135.89552162)
\closepath
}
}
{
\newrgbcolor{curcolor}{0 0 0}
\pscustom[linestyle=none,fillstyle=solid,fillcolor=curcolor]
{
\newpath
\moveto(240.43872533,117.87529987)
\lineto(242.34433504,117.87529987)
\lineto(242.34433504,124.43460909)
\lineto(249.88653769,124.43460909)
\lineto(249.88653769,117.87529987)
\lineto(251.79214741,117.87529987)
\lineto(251.79214741,132.19746129)
\lineto(249.88653769,132.19746129)
\lineto(249.88653769,126.03933306)
\lineto(242.34433504,126.03933306)
\lineto(242.34433504,132.19746129)
\lineto(240.43872533,132.19746129)
\lineto(240.43872533,117.87529987)
}
}
\end{pspicture}

\caption{Cilindro, disco y esfera}
\label{objetos}
\end{figure}

\section{Registro de datos}
\subsection{Péndulo}
Tiempo de diez oscilaciones: \\
\begin{center}
\begin{tabular}{|c|>{\centering}m{3.0cm}<{\centering}|}
\hline
\textbf{n} & \textbf{t[s]} \tabularnewline \hline
1 & \\ \hline
2 & \\ \hline
3 & \\ \hline
4 & \\ \hline
5 & \\ \hline
6 & \\ \hline
7 & \\ \hline
8 & \\ \hline
9 & \\ \hline
10 & \\ \hline
\end{tabular}
\end{center}

\subsection{Cilindro}
Medidas de la longitud y el diámetro del cilindro:
\begin{center}
\begin{tabular}{|c|>{\centering}m{3.0cm}<{\centering}
                  |>{\centering}m{3.0cm}<{\centering}|}
\hline
\textbf{n} & \textbf{L[cm]} & \textbf{D[cm]} \tabularnewline \hline
1 & & \\ \hline
2 & & \\ \hline
3 & & \\ \hline
4 & & \\ \hline
5 & & \\ \hline
6 & & \\
\hline
\end{tabular}
\end{center}
\vspace{0.5cm}
Medida de la masa del cilindro:
\begin{center}
\begin{tabular}{|c|>{\centering}m{3.0cm}<{\centering}|}
\hline
\textbf{n} & \textbf{m[g]} \tabularnewline \hline
1 & \\
\hline
\end{tabular}
\end{center}

\subsection{Disco}
Medidas de la longitud y el diámetro del disco:
\begin{center}
\begin{tabular}{|c|>{\centering}m{3.0cm}<{\centering}
                  |>{\centering}m{3.0cm}<{\centering}|}
\hline
\textbf{n} & \textbf{L[cm]} & \textbf{D[cm]} \tabularnewline \hline
1 & & \\ \hline
2 & & \\ \hline
3 & & \\ \hline
4 & & \\ \hline
5 & & \\ \hline
6 & & \\
\hline
\end{tabular}
\end{center}
\vspace{0.5cm}
Medida de la masa del disco:
\begin{center}
\begin{tabular}{|c|>{\centering}m{3.0cm}<{\centering}|}
\hline
\textbf{n} & \textbf{m[g]} \tabularnewline \hline
1 & \\
\hline
\end{tabular}
\end{center}

\subsection{Esfera}
Medidas del diámetro de la esfera:
\begin{center}
\begin{tabular}{|c|>{\centering}m{3.0cm}<{\centering}|}
\hline
\textbf{n} & \textbf{D[cm]} \tabularnewline \hline
1 & \\ \hline
2 & \\ \hline
3 & \\ \hline
4 & \\ \hline
5 & \\ \hline
6 & \\ \hline
\end{tabular}
\end{center}
\vspace{0.5cm}
Medida de la masa de la esfera:
\begin{center}
\begin{tabular}{|c|>{\centering}m{3.0cm}<{\centering}|}
\hline
\textbf{n} & \textbf{m[g]} \tabularnewline \hline
1 & \\
\hline
\end{tabular}
\end{center}

\section{Cálculos y tablas}

\subsection{Péndulo}
\begin{tabular}{|c|>{\centering}m{2.0cm}<{\centering}
                  |>{\centering}m{2.0cm}<{\centering}
                  |>{\centering}m{2.0cm}<{\centering}|}
\hline
\textbf{$n$} & \textbf{$t[s]$}
             & \textbf{$d_i[s]$}
             & \textbf{$d_i^2[s^2]$} \tabularnewline \hline
1 & & & \\ \hline
2 & & & \\ \hline
3 & & & \\ \hline
4 & & & \\ \hline
5 & & & \\ \hline
6 & & & \\ \hline
7 & & & \\ \hline
8 & & & \\ \hline
9 & & & \\ \hline
10 & & & \\ \hline
 & & & \\ \hline
\end{tabular}\\

\vspace{1.4cm}
\begin{tabular}{|c|p{2.0cm}|}
\hline
$\bar{t}$ & \\ \hline
$\sigma_t$ & \\ \hline
$P$ & \\ \hline
$e_t$ & \\ \hline
\end{tabular}\\

\vspace{1.4cm}
\begin{tabular}{|c|p{2.0cm}|}
\hline
\multicolumn{2}{|c|}{Resultados de la medición} \\ \hline
$L$ & \\ \hline
$t$ & \\ \hline
\end{tabular}

\subsection{Cilindro}
\begin{tabular}{|c|>{\centering}m{2.0cm}<{\centering}
                  |>{\centering}m{2.0cm}<{\centering}
                  |>{\centering}m{2.0cm}<{\centering}|}
\hline
\textbf{$n$} & \textbf{$H[cm]$}
             & \textbf{$d_i[cm]$}
             & \textbf{$d_i^2[cm^2]$} \tabularnewline \hline
1 & & & \\ \hline
2 & & & \\ \hline
3 & & & \\ \hline
4 & & & \\ \hline
5 & & & \\ \hline
6 & & & \\ \hline
 & & & \\ \hline
\end{tabular}\\

\vspace{0.5cm}
\begin{tabular}{|c|p{2.0cm}|}
\hline
$\bar{H}$ & \\ \hline
$\sigma_H$ & \\ \hline
$P$ & \\ \hline
$e_H$ & \\ \hline
\end{tabular}\\

\begin{tabular}{|c|>{\centering}m{2.0cm}<{\centering}
                  |>{\centering}m{2.0cm}<{\centering}
                  |>{\centering}m{2.0cm}<{\centering}|}
\hline
\textbf{$n$} & \textbf{$D[cm]$}
             & \textbf{$d_i[cm]$}
             & \textbf{$d_i^2[cm^2]$} \tabularnewline \hline
1 & & & \\ \hline
2 & & & \\ \hline
3 & & & \\ \hline
4 & & & \\ \hline
5 & & & \\ \hline
6 & & & \\ \hline
 & & & \\ \hline
\end{tabular}\\

\vspace{0.5cm}
\begin{tabular}{|c|p{2.0cm}|}
\hline
$\bar{D}$ & \\ \hline
$\sigma_D$ & \\ \hline
$P$ & \\ \hline
$e_D$ & \\ \hline
\end{tabular}\\

\vspace{1.4cm}
\begin{tabular}{|c|p{2.0cm}|}
\hline
\multicolumn{2}{|c|}{Resultados de la medición} \\ \hline
$H$ & \\ \hline
$D$ & \\ \hline
$m$ & \\ \hline
\end{tabular}

\subsection{Disco}
\begin{tabular}{|c|>{\centering}m{2.0cm}<{\centering}
                  |>{\centering}m{2.0cm}<{\centering}
                  |>{\centering}m{2.0cm}<{\centering}|}
\hline
\textbf{$n$} & \textbf{$H[cm]$}
             & \textbf{$d_i[cm]$}
             & \textbf{$d_i^2[cm^2]$} \tabularnewline \hline
1 & & & \\ \hline
2 & & & \\ \hline
3 & & & \\ \hline
4 & & & \\ \hline
5 & & & \\ \hline
6 & & & \\ \hline
 & & & \\ \hline
\end{tabular}\\

\vspace{0.5cm}
\begin{tabular}{|c|p{2.0cm}|}
\hline
$\bar{H}$ & \\ \hline
$\sigma_H$ & \\ \hline
$P$ & \\ \hline
$e_H$ & \\ \hline
\end{tabular}\\

\begin{tabular}{|c|>{\centering}m{2.0cm}<{\centering}
                  |>{\centering}m{2.0cm}<{\centering}
                  |>{\centering}m{2.0cm}<{\centering}|}
\hline
\textbf{$n$} & \textbf{$D[cm]$}
             & \textbf{$d_i[cm]$}
             & \textbf{$d_i^2[cm^2]$} \tabularnewline \hline
1 & & & \\ \hline
2 & & & \\ \hline
3 & & & \\ \hline
4 & & & \\ \hline
5 & & & \\ \hline
6 & & & \\ \hline
 & & & \\ \hline
\end{tabular}\\

\vspace{0.5cm}
\begin{tabular}{|c|p{2.0cm}|}
\hline
$\bar{D}$ & \\ \hline
$\sigma_D$ & \\ \hline
$P$ & \\ \hline
$e_D$ & \\ \hline
\end{tabular}\\

\vspace{1.4cm}
\begin{tabular}{|c|p{2.0cm}|}
\hline
\multicolumn{2}{|c|}{Resultados de la medición} \\ \hline
$H$ & \\ \hline
$D$ & \\ \hline
$m$ & \\ \hline
\end{tabular}

\subsection{Esfera}
\begin{tabular}{|c|>{\centering}m{2.0cm}<{\centering}
                  |>{\centering}m{2.0cm}<{\centering}
                  |>{\centering}m{2.0cm}<{\centering}|}
\hline
\textbf{$n$} & \textbf{$D[cm]$}
             & \textbf{$d_i[cm]$}
             & \textbf{$d_i^2[cm^2]$} \tabularnewline \hline
1 & & & \\ \hline
2 & & & \\ \hline
3 & & & \\ \hline
4 & & & \\ \hline
5 & & & \\ \hline
6 & & & \\ \hline
 & & & \\ \hline
\end{tabular}\\

\vspace{0.5cm}
\begin{tabular}{|c|p{2.0cm}|}
\hline
$\bar{D}$ & \\ \hline
$\sigma_D$ & \\ \hline
$P$ & \\ \hline
$e_D$ & \\ \hline
\end{tabular}\\

\vspace{1.4cm}
\begin{tabular}{|c|p{2.0cm}|}
\hline
\multicolumn{2}{|c|}{Resultados de la medición} \\ \hline
$D$ & \\ \hline
$m$ & \\ \hline
\end{tabular}

\subsection{Resumen de mediciones}
A continuación se resumen las medidas obtenidas:

\begin{center}
\begin{tabular}{|c|>{\centering}m{2.0cm}<{\centering}
                  |>{\centering}m{2.0cm}<{\centering}
                  |>{\centering}m{2.0cm}<{\centering}|}
\hline
\textbf{Objeto} & \textbf{$H[cm]$}
             & \textbf{$D[cm]$}
             & \textbf{$m[g]$} \tabularnewline \hline
\textbf{Cilindro} & & & \\ \hline
\textbf{Disco} & & & \\ \hline
\textbf{Esfera} & & & \\ \hline
\end{tabular}\\
\end{center}

\section{Conclusiones}
Se han obtenido medidas directas a partir de la medición con instrumentos, se
ha notado la importancia del manejo de la precisión en la toma de muestras,
ademas se ha visto que las herramientas estadísticas brindan una ayuda vital a
la tarea en laboratorio.

\section{Referencias bibliográficas}
\begin{itemize}
\item Fisicanet \\
http://fisicanet.com.ar/física/mediciones/ap01\_errores.php
\item Desviación típica \\
https://es.wikipedia.org/wiki/Desviación\_típica
\end{itemize}

\section{Respuestas al cuestionario}
\begin{enumerate}
    \item ¿Qué es la precisión de un instrumento?
        \vspace{2.0cm}
    \item ¿Qué errores sistemáticos detecto en el proceso de medición?
        \vspace{2.0cm}
    \item ¿Qué criterio utilizo para estimar el error de una medida única?
        \vspace{2.0cm}
    \item ¿Qué criterio utilizo para estimar el error de una serie de medidas?
        \vspace{2.0cm}
    \item En una serie de medidas, ¿Para qué tipo de distribución el valor
    representativo esta dado por la media aritmética?
        \vspace{2.0cm}
    \item ¿Qué mide el parámetro $\sigma_{n-1}$?
        \vspace{2.0cm}
    \item ¿Qué mide el parámetro $\sigma_x$?
        \vspace{2.0cm}
\end{enumerate}

\end{document}
