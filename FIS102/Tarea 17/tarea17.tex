\documentclass[letter,11pt]{article}

\usepackage[spanish,es-nodecimaldot]{babel}
\usepackage[utf8]{inputenc}

\usepackage{lmodern}
\usepackage[T1]{fontenc}
\usepackage{textcomp}

\usepackage{framed}
\usepackage[svgnames]{xcolor}
\colorlet{shadecolor}{Gainsboro!50}

\usepackage[shortlabels]{enumitem}
\usepackage{graphicx}
\usepackage{pstricks}

\usepackage{anysize}
\marginsize{3cm}{2cm}{2cm}{3cm}

\usepackage{siunitx}
\usepackage{amsmath}
\usepackage{array}
\usepackage{alltt}

\usepackage{fancyhdr}
\usepackage{lastpage}
\pagestyle{fancy}
\fancyhf{}
\fancyhead[LE,RO]{Física Básica II}
\fancyfoot[CO,CE]{\thepage\ de \pageref{LastPage}}

\special{papersize=215.9mm,279.4mm}

\usepackage[
    pdfauthor={Carlos Eduardo Caballero Burgoa},%
    pdftitle={Física Básica II},%
    pdfsubject={Tarea 17},%
    colorlinks,%
    citecolor=black,%
    filecolor=black,%
    linkcolor=black,%
    urlcolor=black,
    breaklinks]{hyperref}
\usepackage{breakurl}

\newcommand{\blankpage}{
\newpage
\thispagestyle{empty}
\mbox{}
\newpage
}

\renewcommand{\arraystretch}{1.2}

\begin{document}

\begin{center}
    {\Large \bf{Tarea \#17}}
\end{center}

En un péndulo simple de longitud $l = 3[m]$. Si $\psi_0 = 10^\circ$ y
$\Omega_0 = \dot{\psi_0} = -\frac{\pi}{10} [rad/s]$ para $t = 0$, calcular:

\begin{enumerate}[a)]
    \item La frecuencia angular de oscilación, el periodo de oscilación y la
        frecuencia de oscilación.
    \item La ecuación $\psi = \psi(t)$.
    \item La ecuación $\Omega = \dot{\psi} = \dot{\psi}(t)$.
    \item La ecuación $\alpha = \ddot{\psi} = \ddot{\psi}(t)$.
\end{enumerate}

\vspace{0.5cm}
\textbf{Solución:} \\

\textbf{(a)} \\

Sabiendo que:

\begin{equation}
    \ddot{\psi} + \frac{g}{l} \psi = 0
\end{equation}

y comparando con la ecuación de un oscilador armónico simple:

\begin{equation}
    \ddot{x} + \omega^2_0 x = 0
\end{equation}

Obtenemos la frecuencia angular de oscilación ($\omega_0$):

\begin{equation}
    \omega_0 = \sqrt{\frac{g}{l}}= \sqrt{\frac{9.8}{3}} = 1.8074 [rad/s]
\end{equation}

El periodo de oscilación ($T$):

\begin{equation}
    T = \frac{2 \pi}{\omega_0} = 2 \pi \sqrt{\frac{l}{g}} = 2 \pi \sqrt{\frac{3}{9.8}} = 3.4764 [s]
\end{equation}

Y la frecuencia de oscilación ($\nu$):

\begin{equation}
    f = \frac{1}{T} = \frac{1}{3.4764} = 0.2877 [Hz]
\end{equation}

\textbf{(b)} \\

La solución general de un oscilador armónico simple es:

\begin{equation*}
    \psi = A \cdot cos(\omega_o t - \phi)
\end{equation*}

Considerando las condiciones iniciales para $t = 0$:

\begin{equation*}
    \psi_0 = 10^\circ \cdot \frac{\pi rad}{180^\circ} = \frac{\pi}{18} [rad]
\end{equation*}
\begin{equation*}
    \dot{\psi_0} = - \frac{\pi}{10} [rad/s]
\end{equation*}

Obtenemos las siguientes ecuaciones:

\begin{equation}
    \begin{cases}
        \frac{\pi}{18} = A \cdot cos(1.8074 \cdot t - \phi) \\
        \frac{\pi}{10} = 1.8074 \cdot A \cdot sen(1.8074 \cdot t - \phi)
    \end{cases}
\end{equation}

Elevando ambas ecuaciones al cuadrado y sumándolas:

\begin{equation*}
    \frac{\pi^2}{18^2} + \frac{\pi^2}{(10 \cdot 1.8074)^2} = A^2 \cdot cos^2(1.8074 \cdot t - \phi) + A^2 \cdot sen^2(1.8074 \cdot t - \phi)
\end{equation*}
\begin{equation*}
    A^2 = \frac{\pi^2}{18^2} + \frac{\pi^2}{(18.074)^2} = 0.060675
\end{equation*}
\begin{equation*}
    A = \sqrt{0.060675} = 0.2463 [rad]
\end{equation*}

Despejando $\phi$:

\begin{equation*}
    \frac{\pi}{18} = 0.2463 \cdot cos(-\phi)
\end{equation*}
\begin{equation*}
    cos(\phi) = \frac{\pi}{18 \cdot 0.2463} = 0.7086
\end{equation*}
\begin{equation*}
    \phi = acos(0.7086) = 0.7833
\end{equation*}

Por tanto:

\begin{equation*}
    \psi = A \cdot cos(\omega_0 \cdot t - \phi)
\end{equation*}
\begin{equation}
    \psi = 0.2463 \cdot cos(1.8074 \cdot t - 0.7833)
\end{equation}

\textbf{(c)} \\

Derivando la función $\psi$:

\begin{equation*}
    \dot{\psi} = - A \omega_0 \cdot sen(\omega_0 \cdot t - \phi)
\end{equation*}
\begin{equation*}
    \dot{\psi} = - 0.2463 \cdot 1.8074 \cdot sen(1.8074 \cdot t - 0.7833)
\end{equation*}
\begin{equation}
    \Omega = \dot{\psi} = - 0.4452 \cdot sen(1.8074 \cdot t - 0.7833)
\end{equation}

\textbf{(d)} \\

Derivando la función $\dot{\psi}$:

\begin{equation*}
    \ddot{\psi} = - A \omega^2_0 \cdot cos(\omega_0 \cdot t - \phi)
\end{equation*}
\begin{equation*}
    \ddot{\psi} = - 0.2463 \cdot (1.8074)^2 \cdot cos(1.8074 \cdot t - 0.7833)
\end{equation*}
\begin{equation}
    \alpha = \ddot{\psi} = - 0.8047 \cdot cos(1.8074 \cdot t - 0.7833)
\end{equation}

\end{document}

